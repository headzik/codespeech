\chapter{Introduction}
\label{cha:Introduction}

% description of the problem, why is it important to address it, what solutions I proposed and how well did it work

This thesis deals with a CodeSpeech project - an Eclipse IDE plugin whose goal is to present an alternative for standard means of programming, which is nowadays exclusively done via mouse and keyboard. The prototype allows using programmer's own voice in order to create basic programming structures in Java language and perform basic operations in the integrated development environment. The paper starts with an explanation of motivation standing behind the idea of introduction of such tool and describes the problem that it tries to solve. Then the document continues with a presentation of a whole thinking, creative and practical process of architecture design and implementation phases, followed by system's evaluation and discussion of results to finally finish with a summary of what has been done, what are the lessons learned throughout the whole experience, stating arising conclusions and potential enhancement possibilities to be performed in the future.

\section{Problem}
Computer science studies (or related topics), as the name suggests, involve the use of computers. This includes an enormous amount of hours spent sitting in front of a desk. According to research done by Schlossberg et al. on engineering graduate students,
\begin{quote}\begin{english}
    During the first year, 80\%
    used a computer more than 20 hours per week and by the fifth year this had increased to
    95\%. Those reporting greater than 40 hours of computer use per week increased each
    year in graduate school, beginning with 34\% of participants in the first year and
    increasing to 56\% in the final years of study. \cite{Schlossberg2004} 
 \end{english}\end{quote}
Very often during the late period of studies, an internship or a temporary job in the field is taken up by a student which increases the amount of time spent that way even further. 

 Although results of existing studies on direct relation between usage of computer and severe health conditions vary, many of them involve long term computer users that report pain. Schlossberg et al. confirm in their research done on randomly sampled 206 Electrical Engineering and Computer Science graduate students that \textit{hours of computer use per week is associated with an increased risk of persistent or recurrent upper extremity or neck pain}  \cite{Schlossberg2004}. Lassen et al. performed a year lasting study on 6943 computer operators which ended in a positive association between self-reported mouse and keyboard times and elbow and wrist/hand pain \cite{Lassen2004}. Their later study, however, did not confirm that the use of mouse and keyboard are predictors for persistent and severe pain \cite{Lassen2005}. Chang et al. state that their data \textit{suggest a potential dose–response relationship between daily computer usage time and musculoskeletal symptoms} \cite{Chang2007}. Brandt et al. conclude their study with the statement that \textit{mouse use is associated with an increased risk of moderate-to-severe pain in the neck and right
shoulder, and an association with tension neck syndrome is possible} \cite{Brandt2004}.
 
 Many of the problems are suspected to be caused by repetitive movements that working with peripheral devices such as mouse and keyboard requires. These symptoms started to be known under an umbrella term ``Repetitive Strain Injuries'' (RSI). This topic is broadly described by Van Tulder et al. \cite{VanTulder2007} and Yassi \cite{Yassi1997}. Another factor that supposedly has a negative effect on health is sedentary lifestyle, inevitable in any kind of desk work (so also with computers). Tremblay et al. analyse this problem deeper \cite{Tremblay2010}. Van den Heuvel et al. \cite{Van2007} and Walker-Bone et al. \cite{Walker2005} prove that physical injuries can result such negative consequences as increased amount of absence from work as well as decreased productivity at work. Aforementioned issues do not touch only computer scientists, but any profession that is the use of computers on a daily basis such as office workers, graphic designers \etc  Main problems that are frequently reappearing are neck, upper extremity pain (shoulder, elbow, wrist and hand), Carpal tunnel syndrome, lower back pain and similar musculoskeletal symptoms. 
 
\section{Motivation}
The author of this thesis himself fell victim to some of those issues and after many years of studies developed chronic pain in lower back, neck and recently also shoulders. The pain began to be very problematic not only in times of working on a computer, but also in everyday life. That was the main reason for the idea of creating a new way of interaction with computers with emphasis on programming, which is an occupation of the author.

Progress in certain fields of science such as Machine Learning and Big Data opens up new possibilities that previously could only be imagined. Speech Recognition (SR) is one of the most ``recent'' technologies that benefit from these fields of science and start to very popular. Nowadays SR is used by many and it takes many forms. Be it Amazon's Alexa, Apple's Siri, Microsoft's Cortana, Google's Assistant and more, they all allow for voice control over devices. Speech Recognition is being utilized in smartphones, cars and automated households. It is this technology, that the author saw as the one that brings in 
a new opportunity for alternate way of programming. Unfortunately, available commercial tools are not suitable for this task. Those are mainly created to write text in national languages and perhaps to recognize special commands for system control. Programming languages, however, because of their specific syntax cannot be successfully transcribed by them. The grammar of many programming languages requires awkward use of brackets, parentheses, semi-colons and other punctuations. Therefore a need for an adhoc solution appeared, a tool that could solve the problem of awkward code formatting and enable programming by voice.

Such a tool would not only help to prevent the development of upper extremities pain linked to the use of mouse and keyboard, but also free the programmer from the need of sitting in front of the desk. With nowadays broadly available wireless microphones, the potential software developer could move away from a workstation and take more comfortable position, perform stretches, simple exercises or just walk around the room giving them the opportunity for much more active and healthier lifestyle than it has been possible up to this point. 

Going further with this idea, another potential usage was found. There are people with more severe physical conditions than just chronic pain, namely upper body impairments or a paralysis. The state of such people obviously makes it impossible for them to use a standard way of interacting with computers and thus excludes them from taking up a potential occupation as a programmer. Different tools were created to allow disabled people to utilize computers, none yet specifically designed for coding. How great would it be to enable some people with brilliant minds yet  limited in their bodies to take a job as a programmer and to be as productive as completely healthy person. This could improve their lives in many ways. It could give them purpose and a way of greatly contributing to the society by development of the new software. That obviously would lead to the improvement of their material situation, in result making it easier to pay for many specific needs these people have. 

It seems that, even though such a tool might not have a big impact on the world, it still can prove useful to many existing programmers suffering from occupational injuries and at the same time provide a new group of programmers arising from physically impaired people to fill the ever growing needs of the market.

\section{Goal}
The goal of the project is to create a software that will allow programming using voice. The tool should be somehow compatible with an already existing Integrated Development Environment (IDE) and should utilize available speech recognition engine. Creating new, specific SR engine or programming environment from scratch are not the topic of this thesis, but rather combining different, already existing tools and forming a completely new product. 

\section{Overview}
This section gives an overview of the content of this thesis by briefly describing each subsequent chapter and what can be found in it.

The second chapter titled \textit{Related work} focuses on description of several existing studies related to the topic of voice programming. Most of the selected documents present similar programming by voice projects. These projects are briefly presented in this chapter and compared to the subject of this thesis. Additionally, other research that is not based on any system but is still relevant is mentioned.

The third chapter is titled \textit{Tools} and there one can find description of various software that have been used in order to complete the CodeSpeech project. The first section is dedicated to Eclipse IDE, the second to its extension for plugin development, the third one is focused on three different speech recognition technologies: belonging to CMUSphinx family of open-source tools PocketSphinx and Sphinx4, and a commercial Google Cloud Speech-to-Text service. The end of this section deals with performance comparison thereof. The last section describes a parser software called ANTLR.

Right after that there is \textit{Method} chapter, which is a place where functional and non-functional requirements are stated, use cases defined and the design of architecture is presented. 

The \textit{Implementation} chapter as the name suggests is fully devoted to the process of implementation. It describes the structure of the project based on previously defined architecture, presents the most important classes and interfaces, guides through the workflow of a working cycle to finally discuss some of the interesting features and walk through the process of adding new functionality backed by an example.

The next on the list, \textit{Evaluation} chapter deals with the test that was undertaken on the finished prototype. The approach is being described in detail to leave no doubts about how it was performed. The metrics of interest are explained together with the way they were recorded. Then there is a presentation of the results followed by the discussion of the received scores and what do they mean. In the end, a couple of ideas that arose during or after the evaluation as to how the project can be improved are presented.

Finally, the \textit{Conclusions} chapter summarizes what has been done up to this point. It goes through the statement of the initial problem and how the proposed solution potentially solves it. Then it describes a way the goal was achieved with thinking, creative and practical process behind it, as well as the experience gained throughout the course of development. An objective evaluation of the system overall, how well does it serve its purpose and satisfies initial requirements is given. The chapter and therefore the whole thesis concludes with the statement on the future plans for the project and its continuation.

