\chapter{Abstract}

As other computer users, some of the programmers develop Repetitive
Strain Injuries (RSI) or for other reasons have troubles with
using keyboard and mouse. Not to mention people whose disabilities
prevent them from taking up a work as a programmer in the first
place. With Speech Recognition (SR) systems getting better, another
way of interacting with computers opens up. Programming,
however, because of its specific syntax cannot be successfully
done using available, commercial SR tools. Therefore an idea for a new system for programming by voice arose. That gave birth to a prototype called CodeSpeech, developed as an Eclipse IDE plugin for Java programming language created especially for this purpose. This paper deals with the whole process behind its creation, including requirements definition, design of architecture, description of utilized tools, implementation phase and ends with a first evaluation of the initial iteration's product, deriving conclusions, discussing them and proposing certain solutions for future development. The evaluation compares the new programming method using CodeSpeech with the standard way of programming using mouse and keyboard by recreating a
given sample program and measures the completion time, as well as the use of peripheral devices. In addition, Command Error Rate (CER) metric was measured during the test. The conducted experiment showed that the prototype is around four times slower than the regular programming method, it decreases however the time of mouse and keyboard usage to 37,6\%, distance traveled by the cursor to 17,4\%, keystrokes required to 29,9\% and the number of clicks and doubleclicks to 20\% and 5\% respectively. Calculated CER was 29,5\%, meaning there is room for improvement. The result shows that in the current state CodeSpeech is not feasible to replace mouse and keyboard due to the time it requires for program completion, nonetheless it shows that the potential exists, and with further program development a time might come when its performance becomes competitive to the standard means of programming while keeping all of the benefits of contactless programming.

